\documentclass[]{article}
\usepackage{graphicx}
\usepackage[greek,english]{babel}
\usepackage{textgreek}
\usepackage{url}
\usepackage{caption}
\usepackage{subcaption}
\usepackage{amsmath}
\usepackage{enumitem,calc}

\usepackage[backend=biber]{biblatex}

% Title Page
\title{Multimedia systems assignment}
\author{Thodoris Tyrovouzis 9369}

\SetLabelAlign{myparleft}{\parbox[t]\textwidth{#1\par\mbox{}}}


\begin{document}
\maketitle

\section{Introduction}
This document accompanies the code, providing some general guidance for the source code.

The assignment intents to give hands-on experience with implementing a multimedia coding standard.

My implementation follows the specification up to \textbf{\textgreek{Παραδοτέο Επιπέδου 3}}, using simple and readable code. No file contains over 50 lines of code, thanks to the rich libraries of MATLAB.

\section{Source code description}
The source code contains the following utility functions.

\begin{description}[align=myparleft]
	\item[packFrmBitStrm, unpackFrmBitStrm] Pack and unpack coded bitstream frame packets to and from parameters.
	\item[lar, lar\_inv] Transform LPC reflection coefficients to and from log area ratios
	\item[LTP\_gain\_code, LTP\_gain\_decode] Code and decode long term prediction gains $b_j$.
	\item[preproc, postproc] Preprocess and postprocess procedures (paragraphs 3.1.1, 3.1.2, 3.2.4 of the standard).
	\item[acf] Estimates the autocorrelation function from the samples.
\end{description}

The rest of the functions, \textbf{RPE\_frame\_coder}, \textbf{RPE\_frame\_decoder}, \textbf{RPE\_frame\_ST\_coder}, \textbf{RPE\_frame\_ST\_decoder}, \textbf{RPE\_frame\_SLT\_coder}, \textbf{RPE\_frame\_SLT\_decoder} are implemented as described in the assignment description.

The main function to test the codec is the \textbf{encode\_wav(file)} function. It takes a .wav file as an argument and plays the file after it has been coded and decoded. In the project folder, there is already a speech sample, so the project can be tested by running

\begin{center}
	\texttt{encode\_wav('OSR\_us\_000\_0010\_8k.wav')}
\end{center}

on the MATLAB interpreter.

\section{Conclusion}


\printbibliography
\end{document}          
